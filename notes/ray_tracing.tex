\documentclass[12pt]{article}
 
\usepackage[margin=1in]{geometry} 
\usepackage{amsmath,amsthm,amssymb}
\usepackage[spanish]{babel}
\usepackage{tikz}
\usepackage{tikz-cd}
\usetikzlibrary{babel}
\usepackage[utf8]{inputenc}
\usepackage{amsmath}
\usepackage[shortlabels]{enumitem}
\usepackage{mathtools}
\usepackage{xcolor}
\usepackage{float}
\usepackage{url}
\usepackage{tikz-3dplot} 

\newtheorem{theorem}{Teorema}
\newtheorem{lemma}[theorem]{Lema}
\newtheorem{prop}[theorem]{Proposición}
\newtheorem{coro}[theorem]{Corolario}
\newtheorem{conj}[theorem]{Conjetura}
\newtheorem{ejercicio}{Ejercicio}
\newtheorem*{ejercicio*}{Ejercicio}
\theoremstyle{definition}
\newtheorem{definition}[theorem]{Definición}
\newtheorem{example}[theorem]{Ejemplo}
\theoremstyle{remark}
\newtheorem{remark}[theorem]{Nota}
\newtheorem{notacion}[theorem]{Notación}
 
\graphicspath{{img/}}
\decimalpoint

\begin{document}

\tableofcontents

\vspace{1cm}

Las normales siempre apuntan hacia fuera de la superficie. Si en su lugar tomáramos las normales siempre apuntando contrarias al rayo, entonces no podríamos usar el producto vectorial para determinar en qué lado de la superficie estamos. Por tanto, sería necesario guardar esa información.


\section{Configuración de la cámara}

Para poder trazar rayos a través de nuestro \textit{viewport}, primero es necesario definir una cámara, con un origen, apuntando hacia un determinado \textit{viewport}.

\begin{figure}[H]
\centering
\tdplotsetmaincoords{70}{150}
\tdplotsetrotatedcoords{0}{90}{90}
\begin{tikzpicture} [
scale=2,
tdplot_main_coords, 
tdplot_rotated_coords,
axis/.style={->,orange,thin},
rect/.style={black,thin},
rectuv/.style={red,thin,dashed,->}
]

% axis
\pgfmathsetmacro{\ax}{1}
\draw[axis] (0,0,0) -- (\ax,0,0) node[anchor=north east]{$x_+$};
\draw[axis] (0,0,0) -- (0,\ax,0) node[anchor=north west]{$y_+$};
\draw[axis] (0,0,0) -- (0,0,\ax) node[anchor=south]{$z_+$};

% viewport
\def\x{1};
\def\y{0.5};
\def\z{-2};
\draw[rect] (\x,\y,\z) -- (-\x,\y,\z);
\draw[rect] (\x,\y,\z) -- (\x,-\y,\z);
\draw[rect] (-\x,\y,\z) -- (-\x,-\y,\z);
\draw[rect] (\x,-\y,\z) -- (-\x,-\y,\z);
\node[scale=0.7] at (\x+0.65,\y,\z) {$viewport$};

% lineas
\node[scale=0.7] at (-0.25, 0,0.15) {$origen$};
\draw[dashed,-|] (0,0,0) -- (0,0,\z);

% u,v
\draw[rectuv] (-\x,-\y*0.35,\z) -- (\x*0.4,-\y*0.3,\z);
\node[red, scale=0.55] at (\x*0.2,-0.1 ,\z) {$u$};
\draw[rectuv] (\x*0.4,-\y,\z) -- (\x*0.4,-\y*0.35,\z);
\node[red, scale=0.55] at (\x*0.5,-0.3 ,\z) {$v$};
% ray
\draw[thick, green,->] (0,0,0) -- (\x*0.4,-\y*0.35,\z);
\node[scale=0.4] at (\x*0.6,0,\z*0.4) {$ray$};
\end{tikzpicture}
\end{figure}

Esto permite, usando $u$ y $v$, trazar rayos que pasen por todos los píxeles. Cualquier punto del viewport puede ser expresado como una combinación lineal de $u$ y $v$: 
\[
ray=lower\_corner+u+v,
\]
donde $u=+a*i$ y $v=b*j$, siendo $a$ y $b$ el ancho y el alto del viewport, e $i$ y $j$ vectores unitarios en la parte positiva del eje $Y$ y $X$, respectivamente. 

\subsection{Field of view}

Definir el viewport para cada imagen generada puede ser algo engorroso. Por ello, se usan varias definiciones que permiten definir el viewport a partir de un único parámetro denominado \textit{vertifal field of view}, o \textit{vfov}. Previamente, es necesario establecer una relación entre el ancho y el alto del viewport, lo que se conoce como \textit{aspect ratio}. La relación es: $viewport\_width=aspect\_ratio*viewport\_height$.

Para especificar la altura del viewport, es conveniente usar un parámetro que sea medido en ángulos, para poder hacer \textit{zooming} de forma más cómodo. Para ello, se puede usar la siguiente relación matemática:

\begin{figure}[H]
\centering
\tdplotsetmaincoords{70}{150}
\tdplotsetrotatedcoords{0}{90}{90}
\begin{tikzpicture} [
scale=2,
tdplot_main_coords, 
tdplot_rotated_coords,
axis/.style={->,orange,thin},
rect/.style={black,thin},
rectuv/.style={red,thin,dashed,->}
]

% axis
\pgfmathsetmacro{\ax}{1}
\draw[axis] (0,0,0) -- (\ax,0,0) node[anchor=north east]{$x_+$};
\draw[axis] (0,0,0) -- (0,\ax,0) node[anchor=north west]{$y_+$};
\draw[axis] (0,0,0) -- (0,0,\ax) node[anchor=south]{$z_+$};
\draw[->,orange,thin,dashed] (0,0,0) -- (0,0,-2);
% viewport
\def\x{1};
\def\y{0.5};
\def\z{-2};
\draw[rect] (\x,\y,\z) -- (-\x,\y,\z);
\draw[rect] (\x,\y,\z) -- (\x,-\y,\z);
\draw[rect] (-\x,\y,\z) -- (-\x,-\y,\z);
\draw[rect] (\x,-\y,\z) -- (-\x,-\y,\z);
\node[scale=0.7] at (\x+0.65,\y,\z) {$viewport$};


\draw[dashed,-|] (0,0,0) -- (0,\y,\z);
\draw[dashed,-|] (0,0,0) -- (0,-\y,\z);

\draw [domain=-105:-75, dashed] plot (0,{0.5*cos(\x)}, {0.5*sin(\x)});
\node[scale=0.7] at (0,-0.08,-0.6) {$\theta$};
\draw[|-|,red,thin,dashed] (0,0,\z) -- (0,\y,\z);
\node[scale=0.7, red] at (0,\y*0.45,\z+0.08) {$h$};
\end{tikzpicture}
\end{figure}

De la figura se puede deducir fácilmente que $h=\sin\frac{\theta}{2}$. Usando la definición de la tangente de un ángulo, se tiene que $h=\sin\frac{\theta}{2}=\tan\frac{\theta}{2}\cos\frac{\theta}{2}$. Para dejar esa expresión un poco más \textit{simple}, es posible asumir que la distancia del origen de la cámara al centro del viewport (\textit{focal distance}) va a ser siempre 1, luego $h=\tan\frac{\theta}{2}$.

\subsection{Posicionamiento}

En los dos apartados anteriores, la cámara ha estado situada en el origen y mirando siempre al frente. Evindetemente, este no es el caso usual, luego son necesarios parámetros adicionales para implementar esta característica. 

A la posición donde ponemos la cámara (el centro), se le domina \textit{lookfrom}, y al punto dónde mira, \textit{lookat}. También se podría usar directamente una dirección. Además de estos dos puntos, es necesario saber la rotación de la cámara respecto de la recta formada por esos dos puntos. Esto es necesario ya que aunque se sepa en qué dirección apunta la cámara, no es posible saber cómo de inclinada está (por ejemplo, inclinar la cabeza). Para ello, se define un verctor que apunte \textit{hacia arriba} para la cámara. Este vector debe estar en el plano ortogonal a la dirección formada por los dos puntos anteriores y se denotará por \textit{view up} o \textit{vup}:

\begin{figure}[H]
\centering
\tdplotsetmaincoords{70}{150}
\tdplotsetrotatedcoords{0}{90}{90}
\begin{tikzpicture} [
scale=2,
tdplot_main_coords, 
tdplot_rotated_coords,
axis/.style={->,orange,thin},
rect/.style={black,thin},
rectuv/.style={red,thin,dashed,->}
]

% viewport
\def\x{1.5};
\def\y{0.6};
\def\z{0};
\draw[rect] (\x,\y,\z) -- (-\x,\y,\z);
\draw[rect] (\x,\y,\z) -- (\x,-\y,\z);
\draw[rect] (-\x,\y,\z) -- (-\x,-\y,\z);
\draw[rect] (\x,-\y,\z) -- (-\x,-\y,\z);

\node[red] at (0,0,0) {\textbullet};
\node[red] at (0,-1,-3) {\textbullet};
\draw[red,thin,dashed,->] (0,0,0)--(0,-1,-2.96);
\node[scale=0.6] at (0,-1,-2.7) {$lookat$};

\def\d{0.3}
\draw[red,thin,dashed,->] (0,0,0)--(0,\d,0);
\draw[red,thin,dashed,->] (0,0,0)--(0,-\d,0);
\draw[red,thin,dashed,->] (0,0,0)--(\d,0,0);
\draw[red,thin,dashed,->] (0,0,0)--(-\d,0,0);
\draw[red,thin,dashed,->] (0,0,0)--(\d,\d,0);
\draw[red,thin,dashed,->] (0,0,0)--(-\d,\d,0);
\draw[red,thin,dashed,->] (0,0,0)--(-\d,-\d,0);
\draw[red,thin,dashed,->] (0,0,0)--(\d,-\d,0);

\node[scale=0.6] at (0.9,0.34,0.1) {$lookfrom$};
\end{tikzpicture}
\end{figure}

Con estos tres elementos (lookfrom, lookat y vup) ya es posible formar una base ortonormal describiendo la orientación de nuestra cámara. Para ello sólo hace falta tomar como base los vectores $(u,v,w)$ definidos por:
\[
\left\{\begin{array}{lll}
w = \displaystyle\frac{lookfrom-lookat}{|lookfrom-lookat|}\\
u = \displaystyle\frac{vup\times w}{vup\times w}\\
v = \displaystyleº\frac{w\times u}{w\times u}\\
\end{array}
\right.
\]

Consiguiendo algo similar a la siguiente figura:

\begin{figure}[H]
\centering
\tdplotsetmaincoords{70}{150}
\tdplotsetrotatedcoords{0}{90}{90}
\begin{tikzpicture} [
scale=2,
tdplot_main_coords, 
tdplot_rotated_coords,
axis/.style={->,orange,thin},
rect/.style={black,thin},
rectuv/.style={red,thin,dashed,->}
]

% viewport
\def\x{1.5};
\def\y{0.6};
\def\z{0};
\draw[rect] (\x,\y,\z) -- (-\x,\y,\z);
\draw[rect] (\x,\y,\z) -- (\x,-\y,\z);
\draw[rect] (-\x,\y,\z) -- (-\x,-\y,\z);
\draw[rect] (\x,-\y,\z) -- (-\x,-\y,\z);

\node[red] at (0,0,0) {\textbullet};
\node[red] at (0,-1,-3) {\textbullet};
\draw[red,thin,dashed,->] (0,0,0)--(0,-1,-2.96);

\draw[green,thin,dashed,->] (0,0,0)--(0,1*0.25,2.96*0.25) node[anchor=south] {$w$};
\draw[orange,thin,dashed,->] (0,0,0)--(0,1,0) node[anchor=south] {$v$};
\draw[blue,thin,dashed,->] (0,0,0)--(1,0,0) node[anchor=south] {$u$};
\end{tikzpicture}
\end{figure}

\subsection{Depth of field}

\textbf{este apartado no lo entiendo muy bien}. Lo útlimo que va a ser añadido a esta cámara es lo que los fotógrafos llaman \textit{depth of field}. Para ello es necesario simular una lente entre la cámara y el viewport. A partir de ahora, en lugar de ser generados desde un único punto (lookfrom), los rayos serán generados aleatoriamente dentro un círculo con centro \textit{lookfrom} y un diámetro denominado \textit{aperture}, que será otro parámetro de la cámara. Además, definiremos otro parámetro llamado \textit{focus distance}.

\section{Corrección del color}

\subsection{Antialiasing}

El \textit{antialiasing} es un filtro que suaviza los \textit{dientes de sierra} que se crean al intentar pintar líneas curvas. 

\begin{figure}[H]
   \center
  \includegraphics[scale=0.6]{antialiasing.png}
  \caption{Imagen tomada de \cite{antialiasing}}
\end{figure}

Este problema puede ser corregido tomando varias muestras por cada píxel, es decir, lanzar varios rayos por un mismo píxel y luego hacer la media de los colores obtenidos.

\subsection{Corrección gamma}

Es una operación no linear, usada para codificar y decodificar la luminosidad. Es definido por la siguiente fórmula:
\[
V_{out}=AV_{in}^\gamma
\]
Dependiendo del valor de $\gamma$, la luminosidad de la imagen variará:
\begin{itemize}
\item $\gamma < 1$: normalmente denotado por \textit{encoding gamma} o \textit{gamma compression}. Cuánto más pequeño sea $\gamma$, más clara será la imagen resultante.
\item $\gamma > 1$: normalmente denotado por \textit{decoding gamma} o \textit{gamma decompression}. Cuánto más grande sea $\gamma$, más oscura será la imagen resultante.
\end{itemize}

En nuestro programa, tomamos $A=\frac{1}{nsamples}$, donde $nsamples$ es el número de rayos creados por píxel, y $\gamma=\frac{1}{2}$.

\section{Materiales}

\subsection{Materiales difusos/mates}

Los objetos difusos o mates no emiten luz, simplemente toman el color del ambiente en el que se encuentran, alterandolo ligeramente con su propio color. La dirección de la luz reflejada por un material difuso es aleatoria.

\begin{figure}[H]
\centering
\begin{tikzpicture}
\draw [domain=30:150] plot ({3*cos(\x)}, {0.7*sin(\x)});
\draw [domain=0:360] plot ({cos(\x)}, {1.69+sin(\x)});
\draw[red,thin,->] (-3,2)--({3*cos(110)},{0.7*sin(110)})--({cos(210)}, {1.69+sin(210)})--({3*cos(100)},{0.7*sin(100)})--(-1.5,1.3);
\draw[blue,thin,->] (-3.1,1.9)--({3*cos(115)},{0.7*sin(115)})--({cos(190)}, {1.69+sin(190)})--(-1.5,2);
\draw[green,thin,->] (-3.2,1.8)--({3*cos(120)},{0.7*sin(120)})--(-1.2,1.3);
\end{tikzpicture}
\end{figure}


\subsubsection{Elección de rayos}

Los rayos también podrían ser absorvidos en lugar de reflejados. Cuanto más oscura sea la superficie, más probable hay de que sea absorvido.

\begin{figure}[H]
\centering
\begin{tikzpicture}

\def\r{1.15};
\def\l{4};
\def\R{5.15};
\def\a{315};
\draw[red,dashed] (0,0) circle (\r);
\draw [domain=100:160] plot ({\R*cos(\a)+\l*cos(\x)}, {\R*sin(\a)+\l*sin(\x)});
\node[] at ({\r*cos(\a)},{\r*sin(\a)}) {\textbullet};
\node[] at (0,0) {\textbullet};
\node[red] at (-0.5,-0.2) {\textbullet};
\node[red,scale=0.8] at (-0.5,0.1) {$S$};
\node[] at ({\r*cos(\a)+0.2},{\r*sin(\a)-0.1}) {$P$};
\draw[thin,->]({\r*cos(\a)},{\r*sin(\a)})--(0.03,-0.07);
\node[] at (0.5, -0.3) {$n$};
\node[scale=0.7] at (0.5, 0.1) {$P+n$};
\draw[red,thin,->]({\r*cos(\a)},{\r*sin(\a)})--(-0.5,-0.2);

\node[scale=0.7,orange] at (-0.6, -1.6 ) {$ray$};
\draw[orange,thick,->](-0.7,-2)--({\r*cos(\a)},{\r*sin(\a)});
\end{tikzpicture}
\end{figure}

El punto $S$ es eligido de forma aleatoria dentro de la esfere tangente a $P$ de radio unidad (la de centro $P+n$). El nuevo rayo reflejado será el que pasa por el punto $P$ con dirección $S-P$.

\subsubsection{Shadow Acne}

Algunos de los rayos reflejados, debido a las aproximaciones usadas, intersecan el propio objeto que los está reflejando, luego hay que ignorar los valores de $t$ muy pequeños. Por ejemplo, sólo considerar intersecciones a partir de $t=0.001$.

\subsubsection{Reflexión Lambertiana}

Después de leer \cite{lambertian}, no me queda muy claro esto.

\subsection{Materiales metálicos}

En el caso de metales suaves el rayo no es dispersado aleatoriamente, sino que es reflejado con el mismo ángulo de incidencia pero al otro lado de la normal (ver \cite{beam}).

\begin{figure}[H]
\centering
\begin{tikzpicture}
  \draw[line width=1pt](-4,0)--(4,0);
  \draw[line width=1pt,blue,-stealth](0,0)--(2,2) node[anchor=south west]{$r$};
  \draw[line width=1pt,red,-stealth](-2,2)--(0,0) node[anchor=north east]{$v$};
  \draw [domain=45:135] plot ({0.5*cos(\x)}, {0.5*sin(\x)});
  \node[] at (0.25, 0.7) {$\theta$};
  \node[] at (-0.25, 0.7) {$\theta$};
  \draw[line width=1pt,red,-stealth](0,0)--(2,-2) node[anchor=north east]{$v$};
  \draw[line width=1pt,black,dashed,->](0,0)--(0,1.5) node[anchor=south]{$n$};
  \draw[line width=1pt,dashed,->](2,-2)--(2,0) node[anchor=north west]{$b$};
  \draw[line width=1pt,dashed,->](2,0)--(2,2) node[anchor=north west]{$b$};
  \draw[line width=1pt,dashed,|-|](0,0)--(0,-2);
  \node[] at (-0.8, -1) {$|v|\cos\theta$};
\end{tikzpicture}
\caption{Reflexión}
\end{figure}

A partir del dibujo se ve que la dirección del rayo reflejado es $r=v+2b$. El vector $b$ puede ser calculado usando $n$, que es unitario, y el hecho de que $v\cdot n=|v||n|\cos\theta=|v|\cos\theta$. Como $\angle (v,n)> \frac{\pi}{2}$, $dot(v,n)$ es negativo, luego es necesario añadir un signo menos, quedando:
\[
r=v-2dot(v,n)*n
\]

Una vez calculado el vector reflejado, es posible añadir un grado de \textit{fuzziness} al metal sumando un vector aleatorio a $r$. 

\begin{figure}[H]
\centering
\begin{tikzpicture}
  \draw[line width=1pt](-4,0)--(4,0);
  \draw[line width=1pt,blue,-stealth](0,0)--(2,2);
  \node[blue] at (1, 1.3) {$r$};
  \draw[line width=1pt,red,-stealth](-2,2)--(0,0);
  \node[red] at (-1, 1.3) {$v$};
  \draw[line width=1pt,green,-stealth](0,0)--(2.2,1.8);
  \node[green] at (1, 0.6) {$r'$};
  \draw[line width=1pt,dashed,-stealth](2,2)--(2.2,1.8);
  \node[scale=0.6] at (2.2, 2.05) {$p$};
  \draw[dashed] (2,2) circle (0.5);
\end{tikzpicture}
\end{figure}


donde $p\in B(0,1)$, es un vector aleatorio. Cuanto más grande sea el radio de la bola escogida, más borrosa se verá la imagen. Hay que ser precavido con la elección del radio, ya que si es demasiado grande, el rayo reflejado podría volver a intersecar nuestra superficie.

\subsection{Materiales dieléctricos}

Cuando un rayo alcanza un material dieléctrico (como agua, cristal, etc.), el rayo es dividido en dos partes: un rayo reflejado y otro refractado. La refracción es descrita por la \textit{ley de Snell}:
\[
\eta\sin\theta=\eta'\sin\theta'
\]
donde $\theta$ y $\theta'$ son los ángulos, medidos desde la normal, y $\eta$ y $\eta'$ son los coeficientes de reflexión de los medios, respectivamente.

\begin{figure}[H]
\centering
\begin{tikzpicture}
  \draw[line width=1pt](-2.5,0)--(2.5,0);
  \draw[line width=1pt, dashed](0,-2.5)--(0,2.5);
  \draw[line width=1pt,blue,-stealth](-2,2)--(0,0);
  \draw[line width=1pt,blue,-stealth](0,0)--(1,-2);
  \draw [domain=90:135] plot ({0.5*cos(\x)}, {0.5*sin(\x)});
  \draw [domain=270:295] plot ({0.5*cos(\x)}, {0.5*sin(\x)});
  \node[] at (-2, 0.3) {$\eta$};
  \node[] at (-2, -0.3) {$\eta'$};
  \node[] at (-0.3, 0.6) {$\theta$};
  \node[] at (0.25, -0.8) {$\theta'$};
\end{tikzpicture}
\caption{Refracción}
\end{figure}

Para determinar la dirección del rayo refractado es necesario resolver la ley de Snell, averiguando el valor de $\sin\theta'$, es decir, hay que resolver $\sin\theta'=\frac{\eta}{\eta'}\sin\theta$.

\begin{figure}[H]
\centering
\begin{tikzpicture}
  \draw[line width=1pt](-2.5,0)--(2.5,0);
  \draw[line width=1pt, dashed](0,-2.5)--(0,2.5);
  \draw[line width=1pt,blue,-stealth](-2,2)--(0,0);
  \draw[line width=1pt,blue,-stealth](0,0)--(1,-2);
  \draw [domain=90:135] plot ({0.5*cos(\x)}, {0.5*sin(\x)});
  \draw [domain=270:295] plot ({0.5*cos(\x)}, {0.5*sin(\x)});
  \node[] at (-0.3, 0.6) {$\theta$};
  \node[] at (0.25, -0.8) {$\theta'$};
  \node[blue] at (-2, 1.7) {$R$};
  \node[blue] at (1, -2.3) {$R'$};
  \draw[line width=1pt,red,dashed,->](0,0)--(0,-2);
  \node[red,scale=0.8] at (-0.4, -1.2) {$R'_{||}$};
  \draw[line width=1pt,red,dashed,->](0,0)--(1,0);
  \node[red,scale=0.8] at (0.5, 0.3) {$R'_{\perp}$};
  \draw[line width=1pt,red,dashed](1,0)--(1,-2);
  \draw[line width=1pt,red,dashed](0,-2)--(1,-2);
  \draw[line width=1pt,->](0,0)--(0,0.8);
  \draw[line width=1pt,->](0,0)--(0,-0.8);
  \node[] at (0.3, 0.8) {$n$};
  \node[] at (-0.3, -0.6) {$n'$};
\end{tikzpicture}
\end{figure}

Sea $R'$ el rayo refractado y $R$ el rayo incidente. $n$ y $n'$ son las normales a cada lado de la superficie. $R'$ se puede expresar como la suma de dos vectores, uno perpendicular y otro paralelo a $n'$: $R'=R'_{\perp}+R'_{||}$. Ahora hay que determinar esos dos vectores
\begin{itemize}
\item Usando que $\sin\theta=R+\cos\theta n$, se tiene que $R'_{\perp}=\sin\theta'=\frac{\eta}{\eta'}\sin\theta=\frac{\eta}{\eta'}(R+\cos\theta n)$.
\item Asumiendo que $R'$ es unitario, se puede usar el teorema de Pitágoras para calcular el módulo de $R'_{||}$. Como fue asumido que las normales apuntarían hacia afuera, no se puede usar $n'$, sino que hay que usar $n$, pero $n'=-n$. Luego: $R'_{||}=-\sqrt{1-|R'_{\perp}|^2}n$.
\end{itemize}
Ahora queda resolver la ecuación para $\cos\theta$. Suponiendo que $R$ sea unitario, se tiene que $dot(R,n)=|R||n|\cos\theta=\cos\theta$. Como el ángulo entre $R$ y $n$ es superior a 90 grados, su producto escalar será negativo, luego es necesario poner un signo menos, quedando:
\[
\begin{array}{ll}
R'_{\perp}=\frac{\eta}{\eta'}(R+(-R\cdot n)n)\\
R'_{||}=-\sqrt{1-|R'_{\perp}|^2}n
\end{array}
\]

En el procedimiento anterior hay un problema: si el rayo pasa del material con mayor coeficiente de refracción al de menor, entonces no hay una solución real a la ley de Snell. Por ejemplo, suponiendo $\eta=1.5$ (cristal) y $\eta'=1.0$ (aire) se tendría:
\[
\sin\theta'=\frac{1.5}{1.0}\sin\theta=1.5\sin\theta
\]
luego se podría dar el caso de que $1.5\sin\theta>1$, pero $\sin\theta'$ no puede ser mayor que 1. En esta situación, el material dieléctrico no puede refractar rayos, solo reflejarlos. Esto suele ocurrir cuando el rayo se encuentra dentro un material sólido. Este fenómeno se suele llamar \textit{reflexión interna total}.

Para la reflexión usamos la aproximación de Schilick (ver \cite{schlickapproximation}):
\[
R(\theta)=R_0+(1-R_0)(1-cos\theta)^5, \;\;\;\; R_0=\left(\frac{n_1-n_2}{n_1+n_2}\right)^2
\]
siendo $n1$ y $n2$ los coeficientes de refracción de cada medio.

\begin{thebibliography}{9}
\bibitem{first_book}
\url{https://raytracing.github.io/books/RayTracingInOneWeekend.html}
\bibitem{antialiasing}
\url{https://hardzone.es/reportajes/que-es/anti-aliasing-juegos/}
\bibitem{gamma_correction}
\url{https://en.wikipedia.org/wiki/Gamma_correction}
\bibitem{lambertian}
\url{https://en.wikipedia.org/wiki/Lambertian_reflectance}
\bibitem{beam}
\url{https://www.optics4kids.org/what-is-optics/reflection/the-reflection-of-light#:~:text=Polished%20metal%20surfaces%20reflect%20light,the%20metal%20surface%20is%20reflected.&text=The%20law%20of%20reflection%20requires,imaginary%20line%20(dashed%20in%20Fig.}
\bibitem{tikzmanual}
\url{https://www.bu.edu/math/files/2013/08/tikzpgfmanual.pdf}
\bibitem{snellslaw}
\url{https://en.wikipedia.org/wiki/Snell%27s_law}
\bibitem{schlickapproximation}
\url{https://en.wikipedia.org/wiki/Schlick%27s_approximation}
\end{thebibliography}

\end{document}