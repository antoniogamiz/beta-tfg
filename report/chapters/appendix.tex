\chapter*{Apéndices}
\appendix
\addcontentsline{toc}{chapter}{Appendices}
\renewcommand{\thechapter}{\Alph{section}}

\section{Aclaraciones}

\subsection{Recorrer los vértices de un cubo}

Una forma muy común de recorrer los vértices de un cubo usada en esta implementación es anidar tres bucles diferentes. En total, se realizan 8 iteraciones, cada una de ellas dando como resultado el índice o la localización de uno de los vértices.

\begin{minipage}{0.5\textwidth}
 \begin{lstlisting}[basicstyle=\footnotesize]
void print_vertices()
{
    for (int i = 0; i < 2; i++)
      for (int j = 0; j < 2; j++)
          for (int k = 0; k < 2; k++)
            std::cout << i << ' '
                      << j << ' '
                      << k << std::endl;
}
 \end{lstlisting}
\end{minipage}
\begin{minipage}{0.5\textwidth}
 \inputtikz{cubes}
\end{minipage}
Cada iteración devuelve una terna de valores $(x,y,z)$ donde $x,y,z\in\{0,1\}$. Si se considera cada terna como un número binario, $xyz$, se pueden enumerar los vértices de un cubo y ver fácilmente en qué orden son recorridos:

