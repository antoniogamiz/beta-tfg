\chapter*{}

% ======================= PREFACE =======================

\cleardoublepage
\thispagestyle{empty}

\begin{center}
{\large\bfseries Métodos de Monte-Carlo y desarrollo de software de síntesis de imágenes}\\
\end{center}
\begin{center}
Antonio Gámiz Delgado\\
\end{center}

\noindent{\textbf{Palabras clave}: Ray tracing, Monter-Carlo methods, Synthesis of images}\\

\vspace{0.7cm}
\noindent{\textbf{Resumen}}\\

El objetivo de este trabajo es el estudio de métodos de Monte-Carlo que, en muchas ocasiones, proporcionan el único enfoque manejable para la resolución de problemas tanto de tipo determinísticos como estocásticos. Bajo el nombre de métodos de Monte-Carlo se agrupan diferentes técnicas basadas en el muestreo sistemático de variables aleatorias; por lo tanto, las estimaciones que resultan de los procedimientos de Monte-Carlo tienen errores de muestreo asociados, siendo necesario el estudio de técnicas de reduccción de la varianza para evaluar la bondad de las estimaciones.

En concreto, se realizará el análisis, diseño, implementación y pruebas de software de visualización realista de escenarios y modelos 3D usando métodos de Monte-Carlo, usando como base la literatura relacionada y la base de software Open-Source disponible. Se tendrá como objetivo el desarrollo de software eficiente, robusto y portable. Se pondrá especial énfasis en las técnicas conocidas de reducción de varianza y/o ruido.

% ======================= ABSTRACT =======================

\cleardoublepage
\thispagestyle{empty}

\begin{center}
{\large\bfseries Métodos de Monte-Carlo y desarrollo de software de síntesis de imágenes}\\
\end{center}
\begin{center}
Antonio Gámiz Delgado\\
\end{center}

\noindent{\textbf{Palabras clave}: Ray tracing, Monter-Carlo methods, Synthesis of images}\\

\vspace{0.7cm}
\noindent{\textbf{Abstract}}\\

Traducir el texto de arribilla y ponerlo aquí.

% ======================= FORMALITIES =======================

\chapter*{}
\thispagestyle{empty}

\noindent\rule[-1ex]{\textwidth}{2pt}\\[4.5ex]

Yo, \textbf{Antonio Gámiz Delgado}, alumno de la titulación Ingeniería Informática y Matemáticas de la \textbf{Escuela Técnica Superior
de Ingenierías Informática y de Telecomunicación y de la Facultad de Ciencias de la Universidad de Granada}, con DNI 20886578W, autorizo la
ubicación de la siguiente copia de mi Trabajo Fin de Grado en la biblioteca de ambos centros para que pueda ser consultada por las personas que lo deseen.

\vspace{6cm}

\noindent Fdo: Antonio Gámiz Delgado

\vspace{2cm}

\begin{flushright}
Granada a no sé cuando rellenaré la fecha aquí..
\end{flushright}

% ======================= THANKS =======================

\chapter*{Agradecimientos}
\thispagestyle{empty}
\vspace{1cm}

Ya pensaré algo para poner aquí :D.
