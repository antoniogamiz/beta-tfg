\chapter{Números aleatorios}

Este capítulo pretender cubrir los conceptos básicos sobre números aleatorios y varios métodos sobre cómo generarlos.  Además, también se verán varios procedimientos para verificar y comparar el nivel de \textit{aleatoriedad} de cada método usado.

\section{Noción de secuencia aleatoria}

Un número aleatorio es un número generado por un proceso, cuyo resultado es impredecible y que no se puede reproducir posteriormente de forma fiable. Esta definición va bien siempre que se tenga algún tipo de caja \textit{mágica} (generador de números aleatorios) que cumpla esas condiciones.

Sin embargo, si se tuviera un simple número aleatorio, sería imposible verificar su aleatoriedad, es decir, si fue producido por un generador de números aleatorios o no. Por esta razón, es fundamental que se consideren \textit{secuencias de números aleatorios}.

Dada una secuencia finita de números, tampoco es posible verificar si es aleatoria o no. Lo único que se puede hacer es comprobar si tiene propiedades estadísticas comunes con una secuencia aleatoria - como que los números de la secuencia sean equiprobables - pero esto es difícil de comprobar. Por ejemplo, si se considera un generador de números aleatorios del 0 al 9, la secuencia 4 4 4 4 4, parece ser menos aleatoria que 3 9 4 8 6. Para solventar estas dificultades, hay que elegir una definición común de secuencia de números aleatoria, como la propuesta en \cite{SalmeronMorales}:

\begin{definition}
Una secuencia de números aleatorios es una sucesión de variables aleatorias independientes $\{X_1,\ldots X_n\}$ donde $X_i\leadsto\mathcal{U}[0,1)$ para todo $i=1,\ldots,n$.
\end{definition}

\section{Métodos de generación}

\subsection{Métodos manuales}

Cuando los matemáticos empezaron a necesitar números aleatorios para sus investigaciones, no existían todavía ordenadores para producirlos, luego era necesario recurrir a métodos mas artesanales como el lanzamiento de monedas o dados. Evidentemente, la utilidad de estos métodos para fines estadísticos era escasa.

Estos procedimientos, al ser tan engorrosos, promovieron la creación de tablas de números aleatorios, como la publicada por L.H.C. Tippett, con 40000 entradas diferentes, obtenida a partir de registros del censo (ver \cite{Tippett}).

\subsection{Métodos digitales}

Los métodos digitales son algoritmos numéricos que, a partir de una \textit{semilla}, generan secuencias de números de la forma $X_{n+1}=f(X_n)$. La dependencia funcional de $f$ hace evidente que esta secuencia no es realmente aleatoria.

\begin{definition}
Los números de una secuencia creada a través de un proceso determinista (una rutina, un programa, etc.) reciben el nombre de números pseudoaleatorios.
\end{definition}

El primero de estos métodos es el conocido \textit{Método del centro de los cuadrados} (\textit{middle-square method}), propuesto por John Von Neumann en 1946 (ver \cite{von195113}). El método es bastante simple: consiste en tomar un número $n\in\mathbb{N}$, elevarlo al cuadrado, tomar los valores del centro y repetir el proceso (implementación en \ref{square-center-method}). 

El problema de este método (implementación en \ref{square-center-method}) es su rápida convergencia a cero. No obstante, esto puede ser corregido en cierto modo usando una 


%\begin{minted}{python}
%\end{minted}